\section{Overview}

\subsection{Subsequent Possession}

\begin{enumerate}
    \item % TODO
\end{enumerate}

\subsection{Possessory Estates}

\begin{enumerate}
    \item Fee simple absolute.
    \begin{enumerate}
        \item Potentially infinite.
        \item Common law: ``and his heirs.''
        \item Inheritance: first issue, parents, collaterals.
    \end{enumerate}
    \item Fee tail.
    \begin{enumerate}
        \item Descends to grantee's lineal descendants.
        \item Common law: ``and the heirs of his body.''
        \item When it ends, it reverts to the grantor's heirs.
        \item Only five states recognize it. Others convert the language into 
        some kind of life estate or fee simple.
    \end{enumerate}
    \item Life estate.
    \begin{enumerate}
        \item Today, most are created in trust.
        \item Future interest holders can sue to prevent waste.  
        \emph{Woodrick v. Wood.}
        \item If a conveyance is ambiguous, courts prefer to interpret it as 
        creating a fee simple. \emph{White v. Brown.}
    \end{enumerate}
    \item Leasehold estate.
    \item Defeasible estate.
    \begin{enumerate}
        \item Ends prior to its natural endpoint when a specified event 
        occurs.
        \item Used mainly to control behavior or land use.
        \item Three types:
        \begin{enumerate}
            \item \emph{Fee simple determinable}: durational. Ends 
            automatically. Future interest: possibility of reverter.
            \item \emph{Fee simple subject to condition subsequent}: 
            conditional. \emph{May} be cut short. Future interest: right of 
            entry.
            \item \emph{Fee simple subject to executory limitation}: a third 
            party transferee has the right to take possession of the property 
            if conditions are satisfied. Future interest: executory interest.
        \end{enumerate}
    \end{enumerate}
    \item Restraints on alienation.
    \begin{enumerate}
        \item \emph{Disabling}: a grantee cannot transfer his interest.
        \item \emph{Forfeiture}: a grantee loses his interest if he attempts 
        to transfer it.
        \item \emph{Promissory}: a grantee promises not to transfer his 
        interest. Enforceable through contract remedies.
        \item A use restriction creates a valid FSSCS even if it effectively 
        creates an absolute restraint on alienation. \emph{Mountain Brow Lodge 
        v. Toscano.}
    \end{enumerate}
    % TODO: transferability of future interests?
\end{enumerate}

\subsection{Future Interests}

\begin{enumerate}
    \item Remainder, executory interest, reversion, possibility of reverter, 
    right of entry.
    \item Restatement (Third) reforms:
    \begin{enumerate}
        \item All five interests are combined as ``future interests.''
        \item It reduces the traditional categories---indefeasibly vested, 
        vested subject to complete defeasance, vested subject to open, and 
        contingent---to two: vested and contingent.
    \end{enumerate}
    \item Class gifts to to a class that is still open to new members is 
    classified as \textbf{subject to open}---e.g., ``to A for life, and then 
    to B's children.''
    \item \textbf{Future interests in the transferor}:
    \begin{enumerate}
        \item \textbf{Reversion}: grantor's vested interest after transferring 
        a \textbf{vested estate of a lesser quantum}. E.g., when O conveys 
        ``to A for life,'' O has a reversion.
        \item \textbf{Possibility of reverter}: grantor's vested interest 
        after carving out a \textbf{determinable estate of the same quantum}. 
        It almost always arises when carving out a \textbf{FSD} out of an FSA. 
        E.g., when O conveys ``to A so long as used for school purposes,'' O 
        has a possibility of reverted.
        \item \textbf{Right of entry}: grantor's vested interest when he 
        transfers a \textbf{FSSCS}. E.g., O conveys ``to A, but if it ceases 
        to be used for school purposes, O can reenter and retake the 
        premises.''
    \end{enumerate}
    \item \textbf{Future interests in the transferee}:
    \begin{enumerate}
        \item \textbf{Remainder}:
        \begin{enumerate}
            \item \textbf{Vested}: (1) given to an ascertained person and (2) 
            not subject to a condition precedent. ``~.~.~.~waits politely 
            until the termination of the preceding possessory 
            estate~.~.~.''\footnote{Casebook p. 258.}
            \begin{enumerate}
                \item \emph{Indefeasibly vested}: cannot be divested. E.g., 
                ``to A for life, then to B and her heirs.'' B has an 
                indefeasibly vested remainder.
                \item \emph{Subject to divestment}: vested, but can be 
                divested if an event happens. E.g., ``to A for life, then to B 
                and her heirs, but if B does not survive A, to C and his 
                heirs.''\footnote{Casebook p. 261 example 8.} B has a vested 
                remainder subject to divestment.
                \item \emph{Vested subject to open} (or \emph{subject to 
                partial divestment}): later-born children can share in the 
                gift. E.g., ``to A for life, then to A's children and their 
                heirs.'' At the time of the conveyance, A has one child, B. B 
                has a vested remainder subject to open because A could have 
                more children before she dies.
            \end{enumerate}
            \item \textbf{Contingent}: (1) given to an unascertained person or 
            (2) contingent upon an event other than the natural termination of 
            the preceding estates. E.g., ``to A for life, then to the heirs of 
            B.'' The remainder is contingent because the heirs of B cannot be 
            ascertained until B dies.\footnote{Casebook p. 260 example 5.}
        \end{enumerate}
        \item \textbf{Executory interest}: divests or cuts short an interest 
        in another transferee (\emph{shifting}) or divest the transferor in 
        the future (\emph{springing}).
        \begin{enumerate}
            \item E.g., ``to A and his heirs, but if A dies without issue 
            surviving him, to B and his heirs.'' B has an executory interest, 
            which can only become possessory by divesting A.\footnote{Casebook 
            p. 268.}
            \item Vests automatically, like a possibility of reverter, but 
            \emph{not} like a right of entry.
        \end{enumerate}
        \item \textbf{Executory interest vs. remainder}: ``The difference 
        between taking possession as soon as the prior estate ends and 
        divesting the prior estate is the essential difference between a 
        remainder and an executory interest.''\footnote{Casebook p. 259.}
    \end{enumerate}
    \item \textbf{Trust}: the trustee holds legal title to the trust property 
    and manages it for the benefit of the beneficiaries, who hold equitable 
    title. Trusts can be \textbf{inalienable} and \textbf{beyond the reach of 
    creditors}. \emph{Broadway Natl. Bank v. Adams} and Gray, ``Restraints on 
    Alienation of Property.'' This system can perpetuate wealth inequality, 
    but it can also protect indigent trustees.
    \item \textbf{Rules furthering marketability by destroying contingent 
    future interests}:
    \begin{enumerate}
        \item \textbf{Doctrine of destructability of contingent remainders}:
        \begin{enumerate}
            \item ``A legal remainder in land is destroyed if it does not vest 
            at or before the termination of the preceding freehold 
            estate.''\footnote{Casebook p. 281.}
            \item E.g., O conveys ``to A for life, then to B if B reaches 
            21.'' If A dies before B is 21, B's interest is destroyed.
            \item \textbf{Merger}: ``if the life estate and the next vested 
            estate in fee simple come into the hands of one person, the lesser 
            estate is merged into the larger.''\footnote{Casebook p. 282.}
            \begin{enumerate}
                \item E.g., O conveys to A for life, remainder to B. If A 
                conveys her life estate to B, the life estate merges into B's 
                remainder, giving B a fee simple.
                \item E.g., O conveys to A for life, and then to B if B 
                survives A.  A conveys his life estate to O. The life estate 
                merges with O's fee simple, giving O a fee simple and 
                destroying B's contingent remainder.
            \end{enumerate}
        \end{enumerate}
        \item \textbf{Doctrine of worthier title}:
        \begin{enumerate}
            \item If O conveys to A for life, and then to O's heirs, the 
            remainder in O's heirs is invalid and O gets a reversion. Plugged 
            a feudal tax loophole; mostly obsolete today.
        \end{enumerate}
        \item \textbf{Rule against perpetuities}:
        \begin{enumerate}
            \item The rule applies only to interests that are \textbf{not 
            vested at the time of conveyance}: contingent remainders, 
            executory interests, and class gifts.
            \item Gray: ``No interest is good unless it must vest, if at all, 
            not later than twenty-one years after some life in being at the 
            creation of the interest.''\footnote{Casebook p. 285.}
            \item Analyzing an RAP problem:
            \begin{enumerate}
                \item Identify the interests.
                \item Will each interest vest within the perpetuity period of 
                \emph{any} \textbf{life in being plus 21 years}? If you can 
                \emph{prove} that it will, then the interest is valid. But if 
                you cannot prove the existence of scenario where it would 
                vest, the interest is invalid.
                \item If one or more interests is invalid, what is the remedy 
                for the violation of the RAP?\footnote{Reader p. 73.} (The 
                typical remedy is to strike the invalid remedy, but see the 
                reforms below.)
            \end{enumerate}
            \item Examples:
            \begin{enumerate}
                \item Valid: O transfers land ``in trust for A to life, then 
                to A's first child to reach 21.'' A is the validating life. 
                The interest in A's first child to reach 21 will necessarily 
                vest prior to A's life plus 21 years. Since you can prove that 
                the interest must vest within this period, the remainder is 
                valid.\footnote{Casebook pp. 286--87.}
                \item Invalid: O transfers land ``in trust to A for life, then 
                to A's first child to reach 25.'' The interest will not 
                necessarily vest before after A's life plus 21 years. Thus, 
                the remainder is invalid.
                \item Invalid: to A and his heirs so long as used for school 
                purposes, and then to B and his heirs. The interest will not 
                necessarily vest or terminate within A or B's lifetimes.
            \end{enumerate}
            \item The RAP does not apply to reversionary interests, including 
            possibilities of reverter. \emph{Brown v. Independent Baptist 
            Church of Woburn}. (But it should---this is a loophole.)
            \item The \emph{Klamath Falls} problem: O conveys ``to A and his 
            heirs so long as used as a library, and then to B and his heirs.'' 
            B's remainder is invalid because it it is not certain to vest or 
            fail within 21 years after the death of A or B. But O can get 
            around this problem by conveying an FSD to A and then transferring 
            the possibility of reverter. This is valid because the RAP does 
            not apply to interests in the transferor (reversion, possibility 
            of reverter, right of entry). \emph{City of Klamath 
            Falls v. Flitcraft}. So maybe the RAP \emph{should} apply to those 
            interests.
            \item (jee v. audley) % TODO
            \item A future interest is invalid under the RAP even if it 
            actually vests within the perpetuities period. \emph{The Symphony 
            Space, Inc. v. Pergola Properties, Inc.} (The USRAP abolished the 
            application of the RAP to options and other commercial 
            transactions.\footnote{Casebook p. 304.})
            \item Justifications:\footnote{See Simes, ``Public Policy and the 
            Dead Hand.''}
            \begin{enumerate}
                \item Increase alienability and productivity.
                \item Limit wealth concentration.
                \item It's socially undesirable for people to have assured 
                incomes because it limits social Darwinism.
            \end{enumerate}
            \item Reforms:
            \begin{enumerate}
                \item \emph{Cy pres}: the court rewrites an instrument to 
                avoid an RAP violation in a way that conforms to the 
                transferor's intent.
                \item \emph{Wait-and-see}: the interest is not invalid if it 
                actually vests within the statutory period. Cf. \emph{Symphony 
                Space} above.
                \item \emph{USRAP}: like wait-and-see but with a fixed 90 year 
                period.
            \end{enumerate}
            \item \textbf{Perpetual trust}: many states have exempted trusts 
            from the RAP.
        \end{enumerate}
    \end{enumerate}
\end{enumerate}

\subsection{Co-Ownership}

\subsubsection{Types of Interests}

\begin{enumerate}
    \item \textbf{Tenancy in common}: separate but undivided interests. No 
    survivorship rights. Unlike at common law, the tenancy in common is 
    favored over the joint tenancy.
    \item \textbf{Joint tenancy}: owners are regarded as a single owner. Both 
    have a right of survivorship. Requires four unities (time, title, 
    interest, possession). Any joint tenant can unilaterally sever the joint 
    tenancy by severing one of the unities.
    \item \textbf{Tenancy by the entirety}: like joint tenancy, but with the 
    fifth unity of marriage.
\end{enumerate}

\subsubsection{Severance}

\begin{enumerate}
    \item % TODO
\end{enumerate}

\subsubsection{Relations among Concurrent Owners}

\paragraph{Right to Possession}

\begin{enumerate}
    \item Each cotenant has an \textbf{equal right to possession of the whole 
    property}.
    \item The one major exception is \textbf{ouster}. Ousted cotenants can 
    recover their pro rata share of the fair rental value of the use by the 
    cotenant in possession. \emph{Spiller v. Mackereth}.
    \item A cotenant (either a joint tenant or tenant in common)  can lease to 
    another without the other cotenants' consent, but the lessee's interest 
    cannot exceed the lessor's interest as a cotenant.  \emph{Swartzbaugh v. 
    Sampson}.
    \item Some argue that it makes more sense today to require cotenants in 
    possession to pay rent to the other cotenants under some circumstances, 
    e.g., if the cotenants acquire the property by devise or intestate 
    succession and the other cotenants are already living somewhere 
    else.\footnote{\emph{Understanding Property} p. 140--41.}
\end{enumerate}

\paragraph{Right to Rents and Profits}

\begin{enumerate}
    \item If a third person pays to rent the land, \textbf{each cotenant is 
    entitled to a pro rata share of rents}.
    \begin{enumerate}
        \item The \textbf{Statute of Anne} provided that ``a tenant in common 
        actually receiving rents, issues and profits might be compelled to 
        account for the excess over his proper share.''\footnote{Reader p. 
        122.}
    \end{enumerate}
    \item If a cotenant refuses to pay, the other cotenants can bring an 
    \textbf{accounting action} to recover their shares.
    \item Cotenants are entitled to pro rata shares of profits from natural 
    resources.
\end{enumerate}

\paragraph{Liability for Mortgage and Tax Payments}

\begin{enumerate}
    \item \textbf{All cotenants are obliged} to pay their share of mortages, 
    taxes, assessments, and other payments that could give rise to a lien on 
    the property.
    \item If one tenant pays more than his share, he can recover the excess in 
    a \textbf{contribution action}.
    \begin{enumerate}
        \item Should the amount a cotenant can collect in contributions be 
        offset by the value of his use of the property? The court in 
        \emph{Baird v. Moore} said no. given the circumstances.
    \end{enumerate}
    \item However, in most states, a cotenant in sole possession cannot 
    recover for these payments unless they exceed the reasonable rental value. 
    So if the cotenant in possession spends \$20,000 per year on mortgage 
    payments, but the fair rental value for the year is \$30,000, the cotenant 
    cannot win contributions.
\end{enumerate}

\paragraph{Liability for Repair and Improvement Costs}

\begin{enumerate}
    \item \textbf{Cotenants cannot recover contributions for repairs or 
    improvements} because (1) cotenants may disagree on the scope and 
    necessity of the work and (2) if the law allowed contribution actions for 
    repairs or maintenance, courts would have to adjudicate minor disputes.
    \begin{enumerate}
        \item The \textbf{\emph{Mastbaum} rule}: ``A tenant in common who is 
        in sole possession of the common property is under a duty to his 
        co-tenants to preserve the property by making needful, ordinary 
        repairs, and paying taxes, mortgage interest and insurance 
        premiums.''\footnote{Reader p. 121.} But the general rule is that the  
        cotenant who made the repairs cannot collect contributions from other 
        cotenants.
        \item Some courts will allow contribution as justice requires. 
        \emph{Baird v. Moore}.
    \end{enumerate}
    \item Upon \textbf{partition or an accounting for rent}, a cotenant can 
    recover credit for the \textbf{excess costs} of repairs (i.e., costs 
    beyond his share). For improvements, 
    courts will try to give the improved portion of the property to the 
    cotenant to paid for it, and if that is not possible, it will award a 
    cotenant a credit for the \textbf{added property value} (but not for the 
    actual expense).
    \begin{enumerate}
        \item Example: A and B are cotenants. A builds a building at a cost of 
        \$10,000. Upon partition, the property is sold for \$55,000, with the 
        land worth \$30,000 and the building worth \$25,000. If they split it 
        down the middle, each would get \$27,500. But A should get the entire 
        value of the building. So, they each get half of the value of the land 
        (\$15,000), and A gets the entire value of the building in addition 
        (\$25,000). In total, A gets \$40,000 and B gets \$15,000.
        \item Courts favor partition in kind over a partition sale if it is 
        practical and serves the parties' interests. \emph{Delfino v.  
        Vealencis}.
    \end{enumerate}
\end{enumerate}

\paragraph{Liability for Waste}

\begin{enumerate}
    \item Cotenants are liable to other cotenants for waste.
    \item Courts treat natural resource profits as sources of income to be 
    divided among cotenants.
\end{enumerate}

\subsection{Marital Interests}

\begin{enumerate}
    \item At common law, a woman moved under \emph{cover} at marriage. Husband 
    and wife became one. ``~.~.~.~the husband had the right of possession to 
    all of the wife's lands during marriage, including land acquired after 
    marriage.''\footnote{Casebook p. 360.}
    \item \textbf{Married Women's Property Acts} removed the disabilities of 
    coverture, giving married women control of their property acquired before 
    or during marriage.
    \item \textbf{Protection from creditors}:  an estate held in tenancy by 
    the entirety is immune from the claims of each spouse's separate 
    creditors. \emph{Sawada v. Endo}.
    % TODO expand with `Problems on Marital Property During Marriage'
    \begin{enumerate}
        \item Most states have laws protecting \textbf{homestead rights}, 
        which puts the family home beyond the reach of creditors.
    \end{enumerate}
    % TODO `The Policy of Exempting a Tenancy by the Entirety from Creditors'
    \item Termination of marriage by \textbf{death}:
    \begin{enumerate}
        \item Common law: % TODO dower, curtesy
        \item \textbf{Elective forced share}: % TODO
    \end{enumerate}
    \item Termination of marriage by \textbf{divorce}:
    \begin{enumerate}
        \item % TODO Termination of Marriage by Divorce
        \item % TODO M.B.A. as non-marital property; + celebrity status
    \end{enumerate}
    \item \textbf{Community property}:
    \begin{enumerate}
        \item Adopted in nine states, plus Alaska as an elective community 
        property state.
        \item Earnings of each spouse are owned equally in undivided shares.
        \item All other property (e.g., acquired before marriage, or acquired 
        by gift, devise, or descent) is separate.
        \item % TODO Community Property Compared with Common Law Concurrent 
        % Interests
        \item % TODO Management of Community Property
        \item % TODO Mixing Community Property with Separate Property Problems
        \item \textbf{Migrating couples}: when couples move, their property 
        retains its status as community property or common law property.
    \end{enumerate}
\end{enumerate}

\subsection{Landlord-Tenant Law}

\begin{enumerate}
    \item Leasehold estates:
    \begin{enumerate}
        \item \textbf{Term of years}: fixed period of time.
        \item \textbf{Periodic tenancy}: continues for succeeding period until 
        the tenant or landlord gives notice.
        \item \textbf{Tenancy at will}: no fixed period.
    \end{enumerate}
    \item Holdovers:
    \begin{enumerate}
        \item Arises when a tenant remains in possession after termination of 
        the tenancy.
        \item The landlord can evict or consent. If the landlord cashes 
        subsequent rent checks, a periodic tenancy usually arises, with the 
        term based on the original lease. \emph{Crechale v. Polles, Inc. v. 
        Smith}.
    \end{enumerate}
    \item A lease is both conveyance and contract. Courts often invoke 
    contract principles.
    \item Delivery of possession:
    \begin{enumerate}
        \item \emph{American rule}: no covenant to deliver possession. 
        \emph{Hannah v. Dusch}
        \item \emph{English rule}: there is a covenant.
    \end{enumerate}
    \item Subleases and assignments:
    \begin{enumerate}
        \item \textbf{Assignment}: conveys the entire term, leaving no 
        interest or reversion in the assignor.
        \item \textbf{Sublease}: transfers the lessee's estate for less than 
        the entire term.
        \item At common law, any transfer for less than the lessee's entire 
        term---even by just one day---created a sublease. Today, courts tend 
        to follow the parties' intentions.
    \end{enumerate}
    \item Tenant who defaults:
    \begin{enumerate}
        \item Landlord's remedies: (1) terminate the lease, (2) find another 
        tenant and hold the original tenant liable for any deficiencies, or 
        (3) let the property remain vacant while collecting rent from the 
        original tenant.
        \item \textbf{Self-help}: the common law rule allowed landlords to 
        retake possession on their own. The modern rule encourages landlords 
        to retake possession through a judicial process---e.g., litigation or 
        summary proceeding.
        \item The modern trend is to require landlords to mitigate by finding 
        a new tenant, but at common law there was no duty to mitigate. 
        \emph{Sommer v. Kridel}.
        % TODO: continue at Whitehorn v. Dickerson
    \end{enumerate}
    \item Quiet enjoyment and constructive eviction:
    \begin{enumerate}
        \item \textbf{Substantial interference} constitutes constructive 
        eviction---e.g., repeatedly flooded basement. \emph{Reste Realty Corp. 
        v. Cooper.}
    \end{enumerate}
    \item \textbf{Illegal leases}: tenants can argue that leases violate 
    regulations, e.g., health codes. Unlike actions for quiet enjoyment and 
    constructive eviction, tenants can withhold rent while fending off 
    eviction actions.
    \item \textbf{Implied warranty of habitability}
    % TODO: expand/revise
    \begin{enumerate}
        \item No common law duty. \textbf{Independence of covenants} meant 
        that the lessee's covenant to pay rent was independent of the 
        landlord's duties.
        \item Today, most jurisdictions recognize an implied warranty of 
        habitability for residential leases. \emph{Green v. Superior Court}.
        \item \textbf{Waiver}: tenants generally cannot waive the implied 
        warranty of habitability. \emph{Knight v. Hallsthammer}. % TODO expand
    \end{enumerate}
    % TODO continue at hilder v. st peter
\end{enumerate}

\subsection{Servitudes}

\begin{enumerate}
    \item % TODO
\end{enumerate}
