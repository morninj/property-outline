\section{Eminent Domain and Regulatory Takings}

\begin{enumerate}
    \item Fifth Amendment: ``~.~.~.~nor shall private property be taken for 
    public use, without just compensation.''\footnote{Casebook p. 1061.}
    \item Regulation: zoning.
    \item Taking: eminent domain.
\end{enumerate}

\subsection{Eminent Domain}

\begin{enumerate}
    \item ``Eminent domain is the power of government to force transfers of 
    property from owners to itself.''\footnote{Casebook p. 1061.}
    \item Rationales:\footnote{Casebook p. 1062.}
    \begin{enumerate}
        \item Sovereign states had original and absolute ownership of 
        property, prior to private ownership.
        \item Royal prerogative in feudal times.
        \item Eminent domain is an inherent attribute of sovereignty.
    \end{enumerate}
    \item Why compensate?
    \begin{enumerate}
        \item Moral imperative.
        \item Guard the propertied classes against redistribution of 
        wealth.
    \end{enumerate}
    \item Efficiency arguments on the power to take and the obligation to 
    compensate: see casebook pp. 1063--65.
\end{enumerate}

\subsection{The Public-Use Puzzle}

\begin{enumerate}
    \item The Takings Clause limits eminent domain to ``public use.'' What 
    counts as public use?
\end{enumerate}

\subsubsection{Economic Development as ``Public Use'': \emph{Kelo v. City of 
New London}}

Economic development is a ``public use.''
 
\begin{enumerate}
    \item The City of New London initiated an economic development plan that 
    required it to take petitioners' property (15 houses in all). The 
    petitioners argued that the taking violated the ``public use '' 
    restriction in the Takings Clause.\footnote{Casebook p. 1067.}
    \item The Superior Court granted a restraining order for some of the 
    properties and denied relief for others.
    \item The Connecticut Supreme Court held that all of the takings were 
    valid.
    \item Justice Stevens:
    \begin{enumerate}
        \item Does a city's taking for the purpose of economic development 
        satisfy the public use requirement?
        \item ``Public use'' should be interpreted broadly as ``public 
        purpose''---i.e., the property need not be available to the entire 
        general public as long as the use confers a public benefit.
        \item \emph{Berman v. Parker}: the Court upheld a redevelopment plan 
        for a blighted area, over the objection of the owner of a store that 
        was not blighted, because the area ``must be planned as a 
        whole.''\footnote{Casebook p. 1069.}
        \item \emph{Hawaii Housing Authority v. Midkiff}: the Court upheld a 
        taking to reduce the concentration of land ownership.
        \item Here, the taking was valid because it was part of an economic 
        plan as a whole.
        \item Economic development is a ``public use.''
    \end{enumerate}
    \item Justice Kennedy, concurring:
    \begin{enumerate}
        \item There might need to be a higher standard of review for takings 
        that favor private parties with only incidental public benefits.
    \end{enumerate}
    \item Justice O'Connor, dissenting:
    \begin{enumerate}
        \item The majority's holding would allow any taking as long as there 
        is an incidental public benefit. This benefits those with 
        disproportionate influence in the political process.
    \end{enumerate}
\end{enumerate}

\subsection{Physical Occupations and Regulatory Takings}
 
\begin{enumerate}
    \item How does a government take property?\footnote{Casebook p. 1081.}
    \begin{enumerate}
        \item Attempt negotiations.
        \item At trial, the government establishes its condemnation authority. 
        The court can grant the government permission to enter and inspect, 
        and it can require a deposit.
        \item Jurisdictions vary on whether a jury is required. If it is, the 
        jury sets the compensation amount.
        \item The government then pays compensation plus interest.
        \item Condemnees cannot recover litigation expenses. They can appeal.
    \end{enumerate}
 \end{enumerate}

\subsubsection{\emph{Nollan v. California Coastal Commission}}

\begin{enumerate}
    \item 
\end{enumerate}
 
