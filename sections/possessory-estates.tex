\section{Possessory Estates}

\begin{enumerate}
    \item \textbf{Freehold estate}: normal feudal tenure.
    \item \textbf{Nonfreehold estate}: lease.
    \item ``It is revolting to have no better reason for a rule than that it 
    was laid down in the time of Henry IV. It is still more revolting if the 
    grounds upon which it was laid down have vanished long since, and the rule 
    simply persists from imitation of the past.''---Holmes.\footnote{Casebook 
    p. 183.}
    \item Estates in land: % TODO
    \begin{enumerate}
        \item Present possessory estates.
        \item Future interests/rule against perpetuities.
    \end{enumerate}
\end{enumerate}

\subsection{Up from Feudalism}

\subsubsection{Tenure}

\begin{enumerate}
    \item Beginning with William the Conqueror's social system (Norman 
    Conquest, 1066). Land tenure was a central feature, in which ``one's 
    position was defined in terms of one's relationship to 
    land.''\footnote{Casebook p. 185.} All were subservient to the king.
    \item \textbf{Tenants in chief} held land in exchange for specific 
    services to the king, usually military, e.g., furnishing knights.
    \item The tenant in chief often provided his services by 
    \textbf{subinfeudation}, who provided services or subinfeudated lower 
    tenants in turn---so, a feudal pyramid developed.
    \begin{enumerate}
        \item King $\rightarrow$ Tenant in chief $\rightarrow$ Mesne lord 
        $\rightarrow$ Tenant in demesne.
        \item Each layer owns the services of the layer below.
    \end{enumerate}
    \item The Domesday Book recorded each tract of land in England and the 
    services by which the land is held.
\end{enumerate}

\subsubsection{Feudal Tenures and Services}

\begin{enumerate}
    \item Three major types of tenure with accompanying services:
    \begin{enumerate}
        \item Military tenure: \textbf{knight service} (providing men to fight 
        for the king, or the money equivalent), \textbf{grand sergeantry} 
        (``splendid court life and pageantry''\footnote{Casebook p. 186.}).
        \item Economic tenure: \textbf{socage} (intended to provide 
        subsistence for overlords; e.g., money, agriculture, maintenance, or 
        other economic services). Every land grant included a service, even if 
        nominal like annual delivery of a midsummer rose.
        \item Religious: \textbf{frankalmoign}.\footnote{Casebook p. 187.}
    \end{enumerate}
    \item There were also unfree tenures granted to villeins (peasants), who 
    initially had no legal protection, although courts eventually recognized 
    their holdings as on an equal footing with others.
\end{enumerate}

\subsubsection{Feudal Incidents}

\begin{enumerate}
    \item \textbf{Incidents}: a tenant's duties and liabilities to his lord.
    \item \textbf{Homage and fealty}: ceremonial allegiance.
    \item \textbf{Aids}: financial demands in emergencies, e.g., paying a 
    ransom to the lord's captors.
    \item \textbf{Forfeiture}: if a tenant breached his oath or refused to 
    perform feudal services, he may forfeit his land to the lord. For high 
    treason, he forfeits his land to the king.
    \item When a tenant dies:
    \begin{enumerate}
        \item 
        \item \textbf{Wardship and marriage}: the lord can 
        hold the land for the tenant's underage heirs. The lord could also 
        sell the heir in marriage.
        \item \textbf{Relief}: the heir has to pay the lord to take control of 
        his inheritance.
        \item \textbf{Escheat}: if a tenant dies without heirs, the land goes 
        to the heirs. 
    \end{enumerate}
\end{enumerate}

\subsubsection{Avoidance of Feudal Incidents}

\begin{enumerate}
    \item Tenants developed ways of evading feudal incidents. Suppose a 
    tenant, T, holds land from L by knight service. T subinfeudated to T1, 
    ``reserving as service one rose at midsummer.''\footnote{Casebook p. 189.} 
    T was still responsible for knight service to L, but the subinfeudation 
    devalued the incidents due to L of warship, marriage, relief, and 
    escheat.
    \item Subinfeudation ended with the Statute Quia Emptores in 
    1290.\footnote{Casebook p. 190.}
\end{enumerate}

\subsubsection{The Decline of Feudalism}

\begin{enumerate}
    \item Quia Emptores prohibited subinfeudation in fee simple, but it 
    required lords to allow tenants to transfer their land to others (with the 
    same obligations due to the lord). It had two major consequences:
    \begin{enumerate}
        \item Established a principle of \textbf{free alienation} of land.
        \item Existing mesne relationships disappeared, so that most land was 
        eventually held directly from the crown.
    \end{enumerate}
    \item Wages rose over time, giving peasants increased independence and 
    legal rights.
\end{enumerate}

\subsection{The Fee Simple}

\begin{enumerate}
    \item Tenants have \textbf{status} as (1) tenant of the fee or (2) tenant 
    for life.
    \item Over time, status became \textbf{estate}, defined by the length of 
    time it may endure.
    \item \textbf{Fee simple} (short for fee simple absolute) can endure 
    forever. (Life estate: for the life of the person; a term of years, for 
    some period of time.)
    \item Today, the estate system makes clear (1) what property is being 
    transfered and (2) what sort of ownership is being transferred, measured 
    in the duration of the transferee's interest.\footnote{Casebook p. 191.}
    \item 
\end{enumerate}

\subsubsection{How the Fee Simple Developed}

\begin{enumerate}
    \item \textbf{Heritability}: under feudalism, land was not owned, but held 
    by the possessor as as tenant of another. The tenant's holding 
    (\textbf{fee} or \textbf{fief}) could not be inherited, although the lord 
    would often give it to the heir on payment of relief. Heritability of land 
    gradually arose as a right over time.
    \item \textbf{Alienability}: By the end of the thirteenth century, Quia Emptores 
    ensured that the fee was freely alienable during the tenant's life.
    \item \textbf{Fee simple estate}:
    \begin{enumerate}
        \item Alienability allowed tenants to pass land to others and their 
        heirs, causing holdings to become \textbf{freehold estates} that were 
        not terminable at the lord's will.\footnote{Casebook p. 193.}
        \item \textbf{Estate in land}: e.g., a fee simple---a legal 
        abstraction that we treat as a physical thing (it can be given, sold, 
        bequeathed; creditors can seize it). Estates ``vest, divest, merge, 
        are destroyed, shift, spring.''\footnote{Casebook p. 193.}
    \end{enumerate}
    \item ``The fee simple absolute is as close to unlimited ownership as our 
    law recognizes. It is the largest estate in terms of duration. It may 
    endure forever.''\footnote{Casebook p. 194.}
\end{enumerate}

\subsubsection{Creation of a Fee Simple}

\begin{enumerate}
    \item At early common law, a fee simple was created by conveying land ``to 
    A and his heirs.''
    \item ``And his heirs'' are \textbf{words of limitation}, meaning that the 
    heirs do not take as ``purchasers,'' i.e., they do not have an interest in 
    the land. (They have an interest only after A's death.)
    \item ``To A'' are \textbf{words of purchase}, meaning A is the grantee.
    \item Adding ``and his heirs'' is no longer necessary to create a fee 
    simple, but lawyers still do it out of habit and caution.
\end{enumerate}

\paragraph{Problems on Fee Simple Creation}

\begin{enumerate}
    \item In 1600, O conveys property ``to A for life, then to B forever.'' 
    What estates do A and B have? If A dies and then B dies, who owns the 
    property? What if the conveyance happens in 2002?\footnote{Casebook p. 194 
    problem 1. See reader p. 36, I and II.} % TODO
    ~\\\\\\\\\\\\\\
    \item O conveys property to ``to A and her heirs.'' A's only child runs up 
    large bills. B's creditors can attach B's property to satisfy their 
    claims. Does B have an interest in the property, reachable by B's 
    creditors? What if A wants to sell the property and take a trip around the 
    world---can B stop her?\footnote{Casebook p. 195 problem 3.} % TODO
    \begin{enumerate}
        \item B does not acquire an interest in the property until A's death. 
        So, B's creditors cannot reach the property, and B cannot stop A from 
        selling the property.
    \end{enumerate}

\end{enumerate}

% \subsection{The Fee Tail}
% 
% \subsection{The Life Estate}
% 
% \subsection{Leasehold Estates}
% 
% \subsection{Defeasible Estates}
